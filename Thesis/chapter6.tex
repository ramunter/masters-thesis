%===================================== CHAP 5 =================================

\chapter{Conclusion}\label{ch:conc}

Motivated by the results from the UBE method\citep{donoghue_2017} this thesis has tried to use bayesian linear regression to drive efficient exploration using thompson sampling. Through a linear toy example it was shown that previous bayesian linear regression RL methods do not properly propagate variance between states leading to an underestimation of variance. Furthermore  it was shown that sampling targets from the posterior and including the regression noise variance as a bayesian parameter proper variance propagation was achieved. 

To generalize this method to more complex environments a generalization to neural networks was created based on the similar \cite{azziz_2018} BDQN. The method showed promising results on some environments but had unstable noise variance estimates resulting is bad results on other environments.

This thesis suggests that further work should focus on making the variance estimate dependent on the environment state. This should both increase the stability of the variance and improve performance on environments that require complex exploration, such as Montezuma's Revenge.

\cleardoublepage